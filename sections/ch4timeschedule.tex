\chapter{Time Schedule}\label{ch:timeschedule}
\newcounter{loopcntr}
\newcommand{\rpt}[2][1]{%
  \forloop{loopcntr}{0}{\value{loopcntr}<#1}{#2}%
}
\newcommand{\on}[1][1]{
  \forloop{loopcntr}{0}{\value{loopcntr}<#1}{&\cellcolor{gray}}
}
\newcommand{\off}[1][1]{
  \forloop{loopcntr}{0}{\value{loopcntr}<#1}{&}
}


% A table with rows containing a pbox of 0.17 textwidth, followed by 12 pboxes
% of 0.01 textwidth. Although this sums to just 0.37 textwidth, there are a
% lot of intercolumn gaps to consider. The 0.17 and 0.01 were fiddled by hand
% to fit this particular example.

\noindent\begin{tabular}{p{0.25\textwidth}*{12}{|p{0.04\textwidth}}|}
% The top line
\textbf{Tasks} & \multicolumn{12}{c|}{\textbf{Weeks}} 
        
           \\
% The second line, with its five years of four quarters
\rpt[1]{& 2 & 4 & 6 & 8 & 10 & 12 & 14 & 16 & 18 & 20 & 22 & 24 }  \\
\hline
% using the on macro to fill in twenty cells as `on'
Research   \on[4] \off[8] \\
\hline
Familiarization of tools       \off[3]   \on[2] \off[7]\\
\hline
Designing  \off[4]  \on[2] \off[6] \\
\hline
% using the on macro followed by the off macro
Coding  \off[5]  \on[5] \off[2]\\
\hline

% using the on macro followed by the off macro
Testing \&\ Debugging  \off[7]  \on[4] \off[1] \\ 
\hline
Documentation \on[12] \\

\hline
\end{tabular}
