\pdfbookmark[]{Abstract}{label:abstractpage}
\begin{abstract}
\noindent
In wireless sensor networks (WSN), there exists incomplete (missing) readings because of many reasons like node failures, communication breakdowns, power problems, environmental conditions like fire, natural disasters. Algorithms that depend on sensor data often assume that the readings are complete. Due to missing data entries in WSN,it prevents using such algorithms. To impute these missing values, we propose a method of using deep autoencoders. The model is tested on FORTH-TRACE Dataset version 1.0 which consists of measurements of participants wearing many Shimmer sensor nodes on different body locations, performing various basic and  postural transitions. The performance is compared using absolute mean error between true and imputed data.  It is found that if there is large number of samples, and strong correlation in between sensor values, causing that data can be represented in lower dimensional subspace and matrix is lower rank, imputation result will be improved.
\\[12pt]
\textbf{Key Words:} 
Data imputation, Wireless Sensor Networks, Deep neural network.
\end{abstract}
